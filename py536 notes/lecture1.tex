\documentclass[12pt]{article}
\usepackage[english]{babel}
\usepackage[utf8x]{inputenc}
\usepackage[T1]{fontenc}
\usepackage{scribe}
\usepackage{listings}

% Color
\usepackage[dvipsnames]{xcolor}

% Hyperef 
\usepackage{hyperref}
\hypersetup{
    colorlinks=true,
    linkcolor=blue,
    filecolor=magenta,      
    urlcolor=cyan,
    pdftitle={Overleaf Example},
    pdfpagemode=FullScreen,
    }
\urlstyle{same}

% Bra-Ket Notation
\usepackage{braket}

% For boxes and such
\usepackage[framemethod=TikZ]{mdframed}
\usepackage{lipsum}
\mdfdefinestyle{MyFrame}{%
    linecolor=blue,
    outerlinewidth=2pt,
    roundcorner=20pt,
    innertopmargin=\baselineskip,
    innerbottommargin=\baselineskip,
    innerrightmargin=20pt,
    innerleftmargin=20pt,
    backgroundcolor=gray!50!white}

\Scribe{Alan Nguyen}
\Lecturer{Alexander Sushkov}
\LectureNumber{1}
\LectureDate{September 2022}
\LectureTitle{Introduction}

\lstset{style=mystyle}

\begin{document}
	\MakeScribeTop

%#############################################################
%#############################################################
%#############################################################
%#############################################################

This lecture introduces the basics of Quantum Physics and Quantum Computing.

\section{Introduction to Quantum Computing}

\subsection{Information is Physical} 

\textbf{\color{Orchid} \underline{Information is Physical.}} 
\begin{itemize}
    \item Bits are encoded in a physical system:
    \begin{itemize}
        \item Magnetized domains in a Hard Disk Drive
        \item Neuron synapses connections in the human brain
    \end{itemize}
    \item Computations are carried out by a physical system.
\end{itemize}

The laws of physics \underline{constrain computability} and principles of computation \underline{constrains phenomena}.

% Here's a citation~\cite{Kar84a}.

\subsection{Landauer's Principle} 

\textbf{Landauer's Principle:} To erase a bit of information, the entropy of environment must increase by at least $\delta S$ = $k_Bln(2)$. \\

\noindent
If an erasure process is isothermal, then the amout of heat equal to $T\delta S$ = $Tk_bln(2)$ must flow out of the system.

\paragraph{A thought experiment: \href{https://en.wikipedia.org/wiki/Maxwell's_demon}{Maxwell's Demon}}. The Maxwell's Demon experiment shows that the existent of such a demon would violates the second law of thermodynamics. Therefore,

\textbf{\color{Orchid} Erasing information $\Longleftrightarrow$ Increase in entropy $\Longleftrightarrow$ Irreversibility}

\subsection{Physics is Quantum}

The main ideas are:
\begin{enumerate}
    \item Quantum randomness exists.
    \begin{itemize}
        \item Classical physics is deterministic: probability reflects the lack of knowledge (certainty) about something.
        \item Quantum measurement is fundamentally stochastic.
    \end{itemize}
    \item Heisenberg Uncertainty principle: $\Delta x \Delta p \ge \frac{1}{2}|\braket{[x, p]}|$ = $\frac{\hbar}{2}$.
    \begin{itemize}
        \item if we measure $x$, $\Delta x \rightarrow{} 0$, and that means $p \xrightarrow{} \infty$. This implies that \textbf{\color{Orchid} measurements disturbed the system}
    \end{itemize}
    \item No cloning theorem: quantum states cannot be copied (or else you will violates the Heisenberg Uncertainty Principle).
    \item Bell's Theorem: Quantum Mechanics cannot be reproduced (simulated) by a classical, probabilistic, and local hidden variable theory. This imples that \textbf{\color{Orchid} quantum systems cannot be classicaly simulated.}

\subsection{Classical and Quantum Computations}
Classical interpretation of Computation:
\begin{itemize}
    \item \textbf{bit}: x = 0 or x = 1
    \item \textbf{measure}: when get the value of x.
    \item \textbf{N bits}: "01011...11010" is a binary number with length N.
    \item \textbf{computation}: $x_i$ = "000...000" $\rightarrow{}$ some gates $\rightarrow{} x_f$ = "010...111"
\end{itemize}

Quantum interpretation of Computation:
\begin{itemize}
    \item \textbf{qubit}: state = $\ket{\psi}$ = $a \ket{0} + b \ket{1}$ where a and b are complex amplitudes, and $\ket{0}$ and $\ket{1}$ are basis vectors.

    \item \textbf{measure}: $\ket{0}$ with probability $|a|^2$ and $\ket{1}$ with probability $|b|^2$. Normalisation of probability means that: $|a|^2 + |b|^2$ = 1.

    \item \textbf{N qubits}: $\ket{\psi}$ = $\sum_{x=0}^{2^N - 1} a_x  \ket{x}$. Normalisation of probability means that $\sum_{x} |a_x|^2 = 1. $

    \item \textbf{computation}: let $\ket{\psi_i}$ = $\ket{000...000} $. If we put this state through some (reversible) quantum circuits, we will have the final state $\ket{\psi_f}$ equal to some Unitary matrix $U$ acting on the initial state: $U \ket{\psi_i}$ = $\sum_{x=0}^{2^N - 1} a_x  \ket{x}$. From here, we can carry out the measurement (irreversible) to get $\ket{x}$ with probability equal to $|a_x|^2$.

\end{itemize}

\section{The Great News about Quantum Computing}
Quantum computing may be a very powerful computational paradigm.
\begin{itemize}
    \item A quantum computer can simulate itself, and thus it can (maybe) efficiently simulate other quantum systems. 
\end{itemize}

Why do you want to use a quantum computing system? Answer: quantum algorithms might be a lot more efficient compare to classical algorithms. There are some quantum algorithms that have no equivalent classical versions of themselves. Examples of quantum algorithms:
\begin{itemize}
    \item Shor's Algorithm
    \item Deutsch Algorithm
    \item Grover's Algorithm
\end{itemize}

There is a minimum amount of qubits and other factors that would affect these algorithms (and also required to make them).
    
\end{enumerate}

\nocite{*}

\bibliographystyle{abbrv}           

\bibliography{bib1}                



%%%%%%%%%%% end of doc
\end{document}
